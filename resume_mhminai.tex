% !TXS template
\documentclass[11pt,a4paper,sans,english]{moderncv}
\moderncvstyle{classic}
\moderncvcolor{blue}
%\nopagenumbers{}
\usepackage[utf8]{inputenc}
\usepackage[scale=0.8,a4paper]{geometry}
\usepackage{babel}
%----------------------------------------------------------------------------------
%            personal data
%----------------------------------------------------------------------------------
\firstname{Mohammad Haris}
\familyname{Minai}
\title{Doctoral Candidate (OB \& HRM)}% \\ Indian Institute of Management Lucknow}
%\address{sdsad}{dasdsa}{dasdas}
\mobile{+91-9044823169}
%\phone{Phone}
%\fax{FAX}
\email{mhminai@iiml.ac.in}
\homepage{www.mhminai.com}
%\extrainfo{extrainfo}
%\photo[64pt][0.2pt]{photo_own.jpg}
%\quote{Organizational Behavior and Human Resource Management}
%

%----------- Page numbers, they don't appear automatically ---------------
\usepackage{lastpage}
\rfoot{\addressfont\itshape\textcolor{gray}{Page \thepage\ of \pageref{LastPage}}}
%-------------------------------------------------------------------------

\begin{document}

%-----       resume       ---------------------------------------------------------
\makecvtitle

\section{Education}
\cventry{2012--present}{Doctoral Program in Management (Ph.D.)}{\href{www.iiml.ac.in}{Indian Institute of Management Lucknow}}{}{}{The Fellow Program in Management (FPM) is a doctoral level program offered by the Indian Institutes of Management. It has two years of course work, which culminates in a dissertation. This is followed by a research phase, which concludes with a doctoral thesis submission.}
\cvlistitem{CGPA for coursework: 8.4/10.0}
\cvlistitem{\textbf{Advisor}: \href{www.shailendrasingh.in}{Dr. Shailendra Singh}}
\cvlistitem{Expected submission: June 2017}
\cvlistitem{Thesis title: \textbf{Individual Innovation: Paradox, dynamism, and leadership}}
\cventry{1998--2012}{Work Experience}{Technical Lead and Applications Engineering Manager in the semiconductor industry.}{Please see "Work Experience" section of vita for details.}{Extensive industry experience, particularly in a leadership role and interfacing with top management.}{}
\cventry{1994--1998}{Bachelors in Electronics Engineering} {\href{http://engg.amu.ac.in/department-electronics-engineering.html}{Aligarh Muslim University}}{Aligarh}{}{A four year bachelors program in electronics engineering.}
\cvlistitem{Dissertation: Development of Finite Impulse Response (FIR) filters in VHDL}
\cvlistitem{CGPA: 8.1/10.0}
\cventry{1992--1994}{Senior Secondary School Certificate}{Aligarh Muslim University}{Aligarh}{India}{Qualifying exam for all university and college studies in India}
\cvlistitem{Major in Physics, Mathematics, and Chemistry}
\cvlistitem{Aggregate percentage: 81\%}
\cventry{1992}{GCE O' Levels}{University of London}{School Examinations board}{}{Equivalent to 11 years of education as per English system}
\cvlistitem{Grade A: Mathematics, Physics, Chemistry, and Biology}
\cvlistitem{Grade B: Pure Mathematics, and English}

\section{Manuscripts}
\cvitem{2017}{\textbf{Minai M. H.} \& Varma A. (2017) Cultural relevance of leadership theories: A critical analysis and propositions. \emph{Dimensions.} 5(2), pp 2--12.}
\cvitem{2016}{\textbf{Minai M. H.}, Singh S., and Varma A. (2016) Approaches to Leading for Innovation In S. Munjal \& S. Kundu (Eds.) \emph{Human Capital and Innovation.} pp 39--72. Palgrave.}
\cvitem{Forthcoming}{Gupta M., \& \textbf{Minai M. H.} (in press) An Empirical Analysis of Forecast Performance of GDP Growth in India. \emph{Global Business Review.}}
\cvitem{Under Review}{\textbf{Minai M. H.}, Jauhari H., Kumar M. and Singh S. Unpacking Transformational Leadership: Dimensional Analysis with Psychological Empowerment. \emph{ABDC A category journal.}}

\newpage{}

\section{Research in Progress}
%% 
%% \cvitem{\textbf{Topic:}}{\textbf{Transformational Leadership, Affective Displays and Follower Thinking Processes.}}
%% \cvitem{\textbf{Details:}}{In this study we explore the interactive effects of leadership, and leader affective displays on follower thinking processes. Using a 2x2 experimental design, participants are randomly assigned to transformational or non-transformational leadership displaying either positive affect or negative affect. The participants are tasked with a divergent thinking task and a convergent thinking task. The order in which the tasks are presented is randomized. Follower persistence (motivation) and performance on the tasks are the dependent variables.}
%% \cvitem{\textbf{Status:}}{Filed as working paper. Presented at Doctoral Consortium. Incorporating feedback for submission to journal.}
%% \hfill   \\ \newpage
\cvitem{\textbf{Topic:}}{\textbf{A Dynamic Dialectical Model of Individual Innovation}}
\cvitem{\textbf{Details:}}{A new model is proposed for individual innovation. It adds dynamism and a dialectical perspective. Current models of innovation do not adequately capture these aspects of innovation. We club activities required for innovation into "convergence seeking" and "divergence seeking", joined by an iterative process. This model helps to explain the inconsistent and sometimes contradictory findings in the literature of individual innovation. A case study approach is planned to be used for initial validation of this model.}
\cvitem{\textbf{Status:}}{Filed as working paper. Presented at Doctoral Consortium and a Conference. Analysis and write-up ongoing.}
\hfill   \\
\cvitem{\textbf{Topic:}}{\textbf{Leading for Innovation: A balancing act}}
\cvitem{\textbf{Details:}}{The role of leadership in innovation is explored. A dynamic dialectical model of individual innovation, developed separately, is sought to be further validated during this research. We theorize about the mechanisms by which leaders have an effect on individual innovation and then formalize the model using computational means. The computational model would be validated by comparing the results of the model to prior literature.}
\cvitem{\textbf{Status:}}{Conceptual write-up done. Published book chapter. Computational model is being validated. Research then needs to be written up and submitted to journal.}

%\section{Published Work}

\section{Working Papers}
\cvitem{2016-17}{\textbf{Minai M. H.}, Sahay Y. P., and Singh S. \emph{Too fit to innovate: When flexibility does not lead to innovative work behaviors.}, IIML WPS2016-17/03}
\cvitem{2016-17}{\textbf{Minai M. H.}, Singh S., and Varma A. \emph{An integrative model of individual innovation: Incorporating paradox and dynamism.}, IIML WPS2016-17/02}
\cvitem{2016-17}{\textbf{Minai M. H.}, Singh S., Varma A., and Bhattacharya A. \emph{The use of affective displays by transformational leaders to enhance divergent and/or convergent thinking.}, IIML WPS2016-17/01}
\cvitem{2014-15}{\textbf{Minai M. H.}, Jauhari H., and Singh S. \emph{Exploring the boundary conditions of the relationship between LMX and Work Engagement.}, IIML WPS2014-15/11}
\cvitem{2014-15}{Jauhari H., \textbf{Minai M. H.}, and Singh S. \emph{The role of Transformational Leadership in Psychologically Empowering frontline service employees.}, IIML WPS 2014-15/02}

\section{Workshops and Consortia}
\cvitem{2015}{\textbf{Minai M. H.}, and Singh S., \emph{The Use of Affective Displays by Transformational Leaders to Enhance Divergent and/or Convergent Thinking.}, Paper presented at the IMR Doctoral Students Conference (IMRDC), Bangalore, India.}
\cvitem{2015}{\textbf{Minai M. H.}, and Singh S., \emph{A dynamic dialectical model of individual innovation.}, Paper presented at the Inaugural PhD Consortium 2015, SJMSOM, IIT Bombay, Mumbai, India.}
\cvitem{2015}{\textbf{Minai M. H.}, \emph{Conference and Workshop on Computational Modeling}, Organized by Ohio University, Ohio, USA.}
\cvitem{2015}{\textbf{Minai M. H.}, Jauhari H., and Singh S., \emph{LMX and Work Engagement: The moderating role of core self-evaluations and value congruence with the organization.}, Paper presented at the Inaugural ISB Annual Organizational Behavior Doctoral Consortium, Hyderabad, India.}

\section{Conferences}
\cvitem{2016}{\textbf{Minai M. H.}, Singh S., and Varma A. \emph{The use of affective displays by transformational leaders to enhance divergent and/or convergent thinking.}, Paper presented at the Annual Conference of European Academy of Management 2016, Paris, France.}
\cvitem{2016}{\textbf{Minai M. H.}, and Singh S., \emph{Developing, formalizing and testing a dynamic dialectical model of individual innovation.}, Paper accepted at the 5th Asian Management Research and Case (AMRC) Conference, Dubai, UAE.}
\cvitem{2016}{\textbf{Minai M. H.}, and Singh S., \emph{Leading for innovation: A balancing act.}, Paper accepted at the 5th Asian Management Research and Case (AMRC) Conference, Dubai, UAE.}
\cvitem{2015}{\textbf{Minai M. H.}, and Singh S., \emph{A dynamic dialectical model of innovation.}, Working Paper presented at the 4th Indian Academy of Management Conference, Noida, India.}
\cvitem{2015}{\textbf{Minai M. H.}, Jauhari H., Kumar M., and Singh S., \emph{Unbundling the Dimensions of Transformational Leadership: Dimensional Relationship with Psychological Empowerment.}, Paper presented at the 4th Indian Academy of Management Conference, Noida, India.}
\cvitem{2015}{\textbf{Minai M. H.}, Jauhari H., and Singh S., \emph{LMX and Work Engagement: The moderating role of core self-evaluations and value congruence with the organization.}, Paper presented at the Annual Meeting of Southern Management Association 2015, Florida, USA.}
\cvitem{2014}{\textbf{Minai M. H.}, Jauhari H., and Singh S., \emph{LMX and Work Engagement: A Study of Personal and Organizational Moderators.}, Paper presented at the XXIV Annual Convention of NAOP 2014, Bhopal, India.}
\cvitem{2014}{Gupta M., and \textbf{Minai M. H.} \emph{An empirical analysis of forecast performance of GDP growth in India.}, Poster presented at the 2nd Pan-IIM World Management Conference.}%, IIM Kozhikode, India.}
\cvitem{2008}{Kafeel M. A., Samad A., and \textbf{Minai M. H.} (2008, March, 29th -- 30th) \emph{Basics of FPGA design in context of the AES algorithm.} Paper presented at National Conference on Emerging Technologies (NCET-08), Lucknow, India.}
\cvitem{2007}{Kafeel M. A., and \textbf{Minai M. H.} (2007, September, 8th and 9th) \emph{FPGA based AES Hardware implementation for secure management of systems.} Paper presented at All India Seminar on Communication Convergence (Institution of Engineers), Lucknow, India.}

%\newpage{}

%\section{Honors, Awards and Achievements}
%\cvitem{2015}{Selected as one of four Organizational Behavior scholars across India to present at inaugural OB conference held at Indian School of Business (ISB), Hyderabad, India.}
%\cvitem{2014}{Stood joint first in Srijan 2014, an HR business case competition held at Tata Institute of Social Sciences (TISS), Mumbai, India.}

\section{Other Activities}
\cvitem{2016}{Invited Reviewer for SMA Annual Meeting 2016.}
\cvitem{2016}{Reviewer for AOM Annual Meeting 2016.}
\cvitem{2015}{Coordinated and conducted reviews for IAM 4th Biennial Conference held at IIM Lucknow.}
%\cvitem{2015}{Reviewer for Pan-IIM World Management Conference held at IIM Indore, India.}
%\cvitem{2015}{Reviewer for SMA Annual Meeting 2015.}
\cvitem{2015}{Creation of course outline for first course in Organizational Behavior for MBA students.}
\cvitem{2015}{Coordinator for interview of short listed doctoral candidates, IIM Lucknow, India.}
%\cvitem{2015}{Reviewer for AOM Annual Meeting 2015.}
%\cvitem{2014}{Reviewer for Pan-IIM World Management Conference held at IIM Kozhikode, India.}
\cvitem{2014}{Organized 5-day workshop for doctoral students on \emph{Social Network Analysis} and \emph{Structural Equation Modelling}, IIM Lucknow, India.}
\cvitem{2014}{Organized 3-day workshop for doctoral students on \emph{Meta-analysis}, IIM Lucknow, India.}
\cvitem{2013}{Took a half hour session on "Leadership and Mentoring" for second year MBA students at IIM Lucknow, India.}
\cvitem{2009}{Invited lecture on \emph{"Hardware accelerators for Network Security"} at workshop on "Network Security and Applications", Institute of Engineers, Women's Polytechnic, Aligarh, India.}
\cvitem{2007 \& 2008}{Technical training on \emph{Verilog Hardware Description Language}, for fresh graduates, Freescale Semiconductors, Noida, India.}

\section{Computer Skills}
\cvdoubleitem{\textbf{Office}}{Word, Excel, PowerPoint}{\textbf{Statistics}}{SPSS, AMOS, LISREL, R}
\cvdoubleitem{\textbf{Writing}}{\LaTeX, JabRef(BibTex), Pandoc}{\textbf{Languages}}{R, C, Verilog, VHDL}

\section{Work Experience}
%\subsection{Academic}
%\cventry{--}{}{}{}{}{}
%\cventry{--}{}{}{}{}{}
%\subsection{Non Academic}
\cventry{2005--2012}{Applications Engineering Manager}{Freescale Semiconductors}{Noida}{India}{Started off as a \emph{Senior IC Design Engineer} in 2005, was promoted to \emph{Design Manager} in 2006. In 2009 I changed business group to \textbf{Applications} as \emph{Applications Engineer} and was promoted to \emph{Applications Engineering Manager} in 2010.}
\cventry{2004--2005}{Lead Engineer}{Texas Instruments}{Bangalore}{India}{}
\cventry{2003--2004}{Senior Design Engineer}{Interra Systems}{Noida}{India}{}
\cventry{2001--2003}{ASIC Design Engineer}{Cogency Semiconductors}{Toronto}{Canada}{}
\cventry{2001--2001}{Emulation Engineer}{Tundra Semiconductor}{Ottawa}{Canada}{}
\cventry{1998--2001}{Team Lead}{STMicroelectronics}{Noida}{India}{Was \emph{Design Engineer} for 9 months and for the rest of the duration I worked as a \emph{Team Lead}.}
%\cventry{2005/07/13--2012/06/01}{Applications Engineering Manager}{Freescale Semiconductors}{Noida}{India}{Started off as a \emph{Senior IC Design Engineer} in 2005, was promoted to \emph{Design Manager} in 2006. In 2009 I changed business group to \textbf{Applications} as \emph{Applications Engineer} and was promoted to \emph{Applications Engineering Manager} in 2010.}
%\cventry{2004/05/10--2005/07/11}{Lead Engineer}{Texas Instruments}{Bangalore}{India}{}
%\cventry{2003/12/15--2004/04/26}{Senior Design Engineer}{Interra Systems}{Noida}{India}{}
%\cventry{2001/10/15--2003/05/21}{ASIC Design Engineer}{Cogency Semiconductors}{Toronto}{Canada}{}
%\cventry{2001/04/02--2001/06/22}{Emulation Engineer}{Tundra Semiconductor}{Ottawa}{Canada}{}
%\cventry{1998/12/21--2001/03/12}{Team Lead}{STMicroelectronics}{Noida}{India}{Was \emph{Design Engineer} for 9 months and for the rest of the duration I worked as a \emph{Team Lead}.}
%\cventry{1998/09/30--1998/12/16}{Service Engineer}{Wipro GE Medical Systems}{Lucknow}{India}{}
 %<year%:columnShift:-13,persistent%>%<year%:columnShift:-11,persistent%>%<Job title%:columnShift:-9,persistent%>%<Employer%:columnShift:-7,persistent%>%<City%:columnShift:-5,persistent%>%<%:columnShift:-3,persistent%>%<General description no longer than 1--2 lines%:columnShift:-1,persistent%>

%\section{Accomplishments}
%\cvdoubleitem{What to put here}{And here}{then here}{and here}

%\section{Personal Details}
%\cvdoubleitem{What to put here}{And here}{then here}{and here}

%\newpage{}

\section{References}
\begin{cvcolumns}
    \cvcolumn{Dr. Shailendra Singh}{Director \\ Professor -- HRM Area\\ Indian Institute of Management, Ranchi\\ \href{mailto:shail@iimranchi.ac.in}{shail@iimranchi.ac.in}}
    \cvcolumn{Dr. Abhijit Bhattacharya}{Dean (Faculty)\\ Professor -- Decision Sciences Area\\ Indian Institute of Management, Lucknow\\ \href{mailto:abhijit@iiml.ac.in}{abhijit@iiml.ac.in}}
\end{cvcolumns}
\begin{cvcolumns}
  \cvcolumn{Dr. Arup Varma}{Professor -- Quinlan Business School\\ Loyola University Chicago\\ \href{mailto:avarma@luc.edu}{avarma@luc.edu}}
\end{cvcolumns}

% Further items that can be used
%\cvitemwithcomment{Item 1 Guest}{Item 2 Guest lecture on “Hardware accelerators for.}{Item 3 Guest}
%\cvlistdoubleitem{Item 1 that continues and continues and continues till the end of time}{Item 2 that continues and continues and continues till the end of time}

\clearpage

\end{document}
